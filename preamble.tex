% Archivo: preamble.tex


% ----------------------------------------------------------------
% -------------------- PLANTILLA PERSONAL ------------------------
% ----------------- MODIFICADA EL 03/08/2025 ---------------------
% ----------------------------------------------------------------


% -------------------------- PAQUETES ----------------------------

\usepackage[a4paper, margin=3cm, top=3cm, bottom=3cm]{geometry} % modificar el tamaño de página
\usepackage[utf8]{inputenc} % Codificación del archivo
\usepackage[spanish]{babel} % Idioma del documento
\usepackage{caption} % Personalizar leyendas
\usepackage{colortbl} % colorear tablas
\usepackage{threeparttable} % Asegúrate de incluir este paquete
\usepackage{subcaption} % Personalizar subfiguras
\spanishdecimal{.} % Cambia la coma por punto en números decimales
\usepackage{arydshln} % hacer líneas punteadas en tablas
\usepackage{graphicx} % insertar imágenes
\usepackage{float} % utilizar [H]
\usepackage{array} % personalizar tablas
\usepackage{url} % incluir enlaces
\usepackage{tablefootnote} % notas al pie en tablas
\usepackage{amsmath} % fórmulas matemáticas
\usepackage{hyperref} % hipervínculos
\usepackage{tabularx} % personalizar tablas
\usepackage{tcolorbox} % crear cajas personalizadas
\tcbuselibrary{breakable} % Permite que las cajas se rompan entre páginas
\usetikzlibrary{babel} % Para compatibilidad de TikZ con babel
\usetikzlibrary{decorations.markings} % Added for arrow markings
\usepackage{listings} % personalizar el estilo de los entornos de código 
\usepackage{xcolor} % personalizar colores
\usepackage{color} % personalizar colores
\usepackage{multicol} % utilizar múltiples columnas
\usepackage{fancyhdr} % personalizar la página
\usepackage{titlesec} % personalizar los capítulos y secciones
\usepackage{soul} % Paquete para tachar texto
\usepackage{pgfplots, pgfplotstable, pgf} % Para hacer gráficas
\usepackage{pgfkeys} % Para hacer gráficas
\usepackage{tikz} % Para hacer gráficas
\usepackage{tikzscale} % Para hacer gráficas
\usepackage{multirow} % Para hacer combinación de filas en tablas
\usepackage{mathtools} % Para hacer operaciones matemáticas
\usepackage{amssymb} % Para hacer símbolos matemáticos
\usepackage{enumitem} % Para hacer listas
\usepackage{adjustbox} % Para ajustar tablas
\usepackage{booktabs} % Para hacer tablas
\usepackage{adjustbox} % Para ajustar tablas
\usepackage{tocloft} % Para personalizar la tabla de contenido
\usepackage{lipsum} % texto de relleno
\usepackage{setspace} % espaciado
\usepackage{colortbl} % colorear tablas
\usepackage[symbol]{footmisc} % Cambiar el símbolo de las notas al pie
\usepackage{eso-pic} % poner imagen de fondo
\usepackage{background} % poner imagen de fondo
\usepackage{pdftexcmds} % Para compatibilidad con PDF
\usepackage{etoolbox} % Para compatibilidad con PDF
\usepackage{tabularx} % Para hacer tablas
\usepackage{cancel} % Para tachar texto en ecuaciones
\usepackage{etoolbox} % Para compatibilidad con PDF
\usepackage{wrapfig} % Para poner figuras al lado del texto
\usetikzlibrary{positioning, shapes, arrows.meta,
    decorations.pathreplacing, intersections, shapes.arrows,
    fadings}
\tcbuselibrary{listingsutf8, skins} % Permite el uso de verbatim % Configuración de tcolorbox para incluir TikZ


% ----------------------------------------------------------------
% ----------------- MODIFICABLES DE PREÁMBULO --------------------
% ----------------------------------------------------------------

% -------------------------- COLORES------------------------------

% Definir una macro para el nombre del color principal y secunda-
% rio:
\newcommand{\maincolorname}{blue}
\newcommand{\secondarycolorname}{\maincolorname}
\newcommand{\tertiarycolorname}{\maincolorname}

% Color de fondo: por defecto, el color principal
\newcommand{\bgcolorname}{\maincolorname}

% ----------------------------------------------------------------
% ---------------- NO MODIFICABLES DE PREÁMBULO ------------------
% ----------------------------------------------------------------


% -------------------------- COLORES------------------------------

% Definir Colores:
    % blue agradable: % #0072B2
    \definecolor{blue}{RGB}{0, 114, 178} 
    % red agradable: % #D55E00
    \definecolor{red}{RGB}{213, 94, 0} 
    % gray agradable:  % #808080
    \definecolor{gray}{RGB}{128, 128, 128}
    % orange agradable: % #E69F00
    \definecolor{orange}{RGB}{230, 159, 0}
    % green agradable: % #00AAA6
    \definecolor{green}{RGB}{0, 170, 166}
    % dark green agradable: % #007642
    \definecolor{darkgreen}{RGB}{0, 118, 66}
    % yellow agradable: % #F0E442
    \definecolor{yellow}{RGB}{240, 228, 66} 
    % purple agradable: % #7100B2
    \definecolor{purple}{RGB}{113, 0, 178}
    % pink agradable: % #CB4EBC
    \definecolor{pink}{RGB}{203, 78, 188}
    % brown agradable: % #8C613C
    \definecolor{brown}{RGB}{140, 97, 60}
    % cyan agradable: % #44D3CB
    \definecolor{cyan}{RGB}{68, 211, 203}



% Main color: To use a defined color
\colorlet{main}{\maincolorname}
\colorlet{secondary}{\secondarycolorname}
\colorlet{tertiary}{\tertiarycolorname}

\colorlet{bgcolor}{\bgcolorname}



% ---------------------- ESTILO DE LISTAS-------------------------
% ------- ESTILO DE LAS ETIQUETAS, REFERENCIAS Y CAPTIONS --------

% Cambiar el color de las etiquetas de referencias y ecuaciones
\usepackage{hyperref}
\hypersetup{
    colorlinks=true,
    linkcolor=main,    % Color de enlaces internos
    citecolor=main,    % Color de las citas
    urlcolor=main      % Color de enlaces externos
}

% Aplicar el color a captions de figuras y tablas
\usepackage{caption}
\captionsetup{
    labelfont={color=main,bf}, % Color y negritas para etiquetas
    textfont={color=main}      % Color del texto de la leyenda
}

% Renombrar tablas e índice de tablas
\renewcommand{\listtablename}{Índice de tablas}
\renewcommand{\tablename}{Tabla}
\captionsetup[table]{name=Tabla}



% ---------------------- ESTILO DE LISTAS-------------------------

% Interlineado del documento
    \singlespacing
    % \onehalfspacing
    % \doublespacing




% ---------------------- ESTILO DE LISTAS-------------------------

% Redefinir estilo de itemize y enumerate
% Cambia el color del símbolo de 'itemize':
\setlist[itemize]{label=\textcolor{main}{\textbullet}} 
% Cambia el color y pone en negrita los números de 'enumerate'
\setlist[enumerate]{label=\textcolor{main}{\bfseries\arabic*.}}



% -------------------- ESTILO DE CAPTIONS ------------------------

% Cambiar tamaño de caption
\captionsetup{font=small}



% -------------------- ESTILO DE FOOTNOTE ------------------------

% COLOR

\makeatletter
% Cambiar el color del símbolo de la footnote
\renewcommand{\@makefnmark}{\textcolor{main}{\textsuperscript{\@thefnmark}}}
% Cambiar el color del texto de la footnote
\renewcommand{\@makefntext}[1]{\noindent\textcolor{main}{\textsuperscript{\@thefnmark}} #1}
\makeatother

% ESTILO

\setlength{\footnotesep}{1em} % Separación vertical entre las notas al pie y el texto principal
\setlength{\skip\footins}{1em} % Separación entre la línea y las notas al pie
\renewcommand{\footnotemargin}{1em} % Margen colgante para simular viñetas

% Configuración del interlineado específico para las notas al pie
\renewcommand{\footnotelayout}{\setstretch{1.2}} % Cambia "1.2" por el valor que desees

% Números en lugar de símbolos
\renewcommand{\thefootnote}{\arabic{footnote}}

% ----------------------- CANCELAR COLOR--------------------------

\let\cancelorigcolor\CancelColor% Just for conveniency...

\newcommand{\CancelTo}[3][]{%
  \ifblank{#1}{}{%
    \renewcommand{\CancelColor}{#1}%
  }
  \cancelto{#2}{#3}% 
}



% -------------- ESTILO DE LOS TÍTULOS DE SECCIÓN ----------------

% section con linea abajo y color 'main':
\titleformat{\section}
{\normalfont\fontsize{14}{15}\bfseries\color{main}}
{\thesection}{1em}{}[{\color{main}\titlerule[0.8pt]}]

% subsection con linea abajo y color 'main':
\titleformat{\subsection}
{\normalfont\fontsize{12}{15}\bfseries\color{secondary}}
{\thesubsection}{1em}{}[{\color{secondary}\titlerule[0.8pt]}]



% ------------- CAMBIAR COLOR TAGS DE ECUACIONES -----------------

% Redefinir \tag para que utilice el color main automáticamente
\let\oldtag\tag
\renewcommand{\tag}[1]{\oldtag{\textcolor{main}{#1}}}



% --------------------- ESTILO DE TEXTO --------------------------

% Configurando el estilo del texto:
% Sin sangría y con espacio entre párrafos:
\setlength{\parindent}{0cm}
\setlength{\parskip}{0.5em}



% -------------------- ESTILOS DE TABLAS -------------------------

% Aumentar el alto de las filas de las tablas
\newcolumntype{P}[1]{>{\arraybackslash}p{#1}}
\newcommand{\ra}[1]{\renewcommand{\arraystretch}{#1}}
\ra{1.5} % Aumenta el alto de las filas de las tablas en un 30%

% Color de linea de tablas:
\arrayrulecolor{main}

% Configuración de la tabla:
\setlength{\fboxsep}{10pt} % Espacio alrededor del contenido
\setlength{\fboxrule}{0pt} % Grosor de la línea de la caja (0pt
% significa "sin línea")

% Hacer que esto aplique en \tcolorbox:
\tcbset{before upper={\parskip=0.5em}}




% --------------------- ESTILO DE CAJAS --------------------------

\pgfplotsset{compat=newest}
\usepgfplotslibrary{fillbetween}


% ----------------- ESTILOS DE HIPERVÍNCULOS ---------------------

% Color to links and underline links:
\hypersetup{
    colorlinks=true,
    linkcolor=main!60!black,
    filecolor=magenta,      
    urlcolor=blue,
    linkbordercolor=blue,
    pdfborderstyle={/S/U/W 1},
}


% ------------ CONFIGURACIÓN DE CAPÍTULOS Y SECCIONES ------------

% subsection de tamaño de texto:


% ---------------------- NUEVOS COMANDOS -------------------------

% Crear línea horizontal:
\newcommand{\HRule}{\noindent\makebox[\linewidth]{\rule{\linewidth}{0.4pt}}} 

% Crear línea horizontal gris:
\newcommand{\HRuleGray}{\noindent\makebox[\linewidth]{\color{gray!40}\rule{\linewidth}{0.4pt}}}

% Definir una macro condicional para verificar si el archivo de fondo
% existe
\newcommand{\setbackgroundimage}{
  \IfFileExists{img/bg/page_background_\bgcolorname.pdf}{%
    % Si el archivo existe, úsalo
    \backgroundsetup{
      scale=1,
      color=black,
      opacity=1,
      angle=0,
      position=current page.center,
      vshift=0pt,
      hshift=0pt,
      contents={\includegraphics[width=\paperwidth, height=\paperheight]{img/bg/page_background_\bgcolorname.pdf}}
    }%
  }{%
    % Si el archivo no existe, usar el fondo por defecto
    \backgroundsetup{
      scale=1,
      color=black,
      opacity=1,
      angle=0,
      position=current page.center,
      vshift=0pt,
      hshift=0pt,
      contents={\includegraphics[width=\paperwidth, height=\paperheight]{img/bg/page_background_default.pdf}}
    }%
  }%
}

% Llamar a la macro para establecer la imagen de fondo
\setbackgroundimage

% ----------------------------------------------------------------

% Personalización del entorno lstlisting:
\lstset{
    literate=%
    {á}{{\'a}}1
    {é}{{\'e}}1
    {í}{{\'i}}1
    {ó}{{\'o}}1
    {ú}{{\'u}}1
    {Á}{{\'A}}1
    {É}{{\'E}}1
    {Í}{{\'I}}1
    {Ó}{{\'O}}1
    {Ú}{{\'U}}1
    {‘}{{'}}1
    {’}{{'}}1
    {─}{{-}}1
    {─}{{-}}1
    {—}{{-}}1
    {–}{{-}}1
    {▇}{{-}}1
    {█}{{-}}1
    {▀}{{-}}1
    {▄}{{-}}1
    {♥}{{-}}1
    {∙}{{-}}1
    {→}{{-}}1
    {←}{{-}}1
    {↑}{{-}}1
    {↓}{{-}}1
    {█}{{-}}1
    {▐}{{-}}1
    {▌}{{-}}1
    {░}{{-}}1
    {▒}{{-}}1
    {█}{{-}}1
    {▓}{{-}}1
    {║}{{-}}1
    {╗}{{-}}1
    {╝}{{-}}1
    {╚}{{-}}1
    {╔}{{-}}1
    {═}{{-}}1
    {╠}{{-}}1
    {╣}{{-}}1
    {╬}{{-}}1
    {«}{{-}}1
    {»}{{-}}1
    {ñ}{{\~n}}1
    {Ñ}{{\~N}}1,
    basicstyle=\ttfamily,
    columns=fullflexible,
    frame=single,
    breaklines=true,
    postbreak=\mbox{\textcolor{red}{$\hookrightarrow$}\space},
    xleftmargin=0.45cm, % Margen izquierdo
    xrightmargin=0.35cm, % Margen derecho
    % Tamaño de letra chico:
    basicstyle=\fontsize{9}{11}\selectfont\ttfamily,
    % Padding interno de la caja de 0.5cm:
        framexleftmargin=0.3cm,
        framexrightmargin=0.2cm,
        framextopmargin=0.2cm,
        framexbottommargin=0.2cm,
    % Background color grsi claro:
    backgroundcolor=\color{lightgray!15},
    % Bordes redondeados
    frameround=tttt,
    % roundcorner=0.1cm,
    % Redondeo de los bordes a 0.1cm:
    % Color de la línea gris:
    rulecolor=\color{gray!60},
    % moredelim=[is][\color{red}\textcolor{blue}]{$}{$}, % Delimitador de colores
}

% Personalizando background color de lstinline:
\lstset{backgroundcolor = \color{gray!5}}

% lstlisting de color de código de R cómo monokai:
\lstdefinestyle{R}{ % Estilo de código de R
    language=R, % Lenguaje
    basicstyle=\ttfamily\footnotesize, % Tamaño de letra y tipo de letra
    stepnumber=1, % Número de línea
    commentstyle=\color{green}, % Color de los comentarios
    keywordstyle=\bfseries\color{blue}, % Color de las palabras clave
    stringstyle=\color{orange}, % Color de las strings
    showstringspaces=false, % No mostrar espacios en las strings
    columns=fullflexible,
    frame=single,
    breaklines=true,
    postbreak=\mbox{\textcolor{red}{$\hookrightarrow$}\space},
    % xleftmargin=1.2cm, % Margen izquierdo
    % xrightmargin=0.5cm, % Margen derecho
    % Tamaño de letra chico:
    basicstyle=\fontsize{9}{11}\selectfont\ttfamily,
    % Padding interno de la caja de 0.5cm:
        framexleftmargin=0.3cm,
        framexrightmargin=0.2cm,
        framextopmargin=0.2cm,
        framexbottommargin=0.2cm,
    % Background color grsi claro:
    backgroundcolor=\color{lightgray!15},
    % Bordes redondeados a 0.1cm:
    frameround=tttt,
    % Color de la línea gris:
    rulecolor=\color{gray!60},
    % Números de línea de color gris:
    numberstyle=\color{gray},
    % Números de línea en tamaño chico:
    numbersep=5pt,
    % Números de línea con padding de 0.5cm:
    numbersep=0.5cm,
}

\lstdefinestyle{userR}{ % Estilo de código de R
    language=R, % Lenguaje
    basicstyle=\ttfamily\footnotesize, % Tamaño de letra y tipo de letra
    stepnumber=1, % Número de línea
    commentstyle=\color{green}, % Color de los comentarios
    keywordstyle=\bfseries\color{blue}, % Color de las palabras clave
    stringstyle=\color{orange}, % Color de las strings
    showstringspaces=false, % No mostrar espacios en las strings
    columns=fullflexible,
    frame=single,
    breaklines=true,
    postbreak=\mbox{\textcolor{red}{$\hookrightarrow$}\space},
    % xleftmargin=1.2cm, % Margen izquierdo
    % xrightmargin=0.5cm, % Margen derecho
    % Tamaño de letra chico:
    basicstyle=\fontsize{9}{11}\selectfont\ttfamily,
    % Padding interno de la caja de 0.5cm:
        framexleftmargin=0.3cm,
        framexrightmargin=0.2cm,
        framextopmargin=0.2cm,
        framexbottommargin=0.2cm,
    % Background color grsi claro:
    backgroundcolor=\color{green!10},
    % Bordes redondeados a 0.1cm:
    frameround=tttt,
    % Color de la línea gris:
    rulecolor=\color{gray!60},
    % Números de línea de color gris:
    numberstyle=\color{gray},
    % Números de línea en tamaño chico:
    numbersep=5pt,
    % Números de línea con padding de 0.5cm:
    numbersep=0.5cm,
}

\lstdefinestyle{output}{
    language=bash,
    stepnumber=1, % Número de línea
    commentstyle=\color{green}, % Color de los comentarios
    keywordstyle=\bfseries\color{blue}, % Color de las palabras clave
    stringstyle=\color{orange}, % Color de las strings
    showstringspaces=false, % No mostrar espacios en las strings
    columns=fullflexible,
    keepspaces=true,
    frame=single,
    breaklines=true,
    postbreak=\mbox{\textcolor{red}{$\hookrightarrow$}\space},
    basicstyle=\fontsize{9}{11}\color{black!60}\selectfont\ttfamily,
        framexleftmargin=0.3cm,
        framexrightmargin=0.2cm,
        framextopmargin=0.2cm,
        framexbottommargin=0.2cm,
    backgroundcolor=\color{black!10},
    frameround=tttt,
    rulecolor=\color{gray!60},
    numberstyle=\color{gray},
    % numbersep=0.5cm,
}


% ---------------- PERSONALIZACIÓN DE LA PÁGINA ------------------

% Personalización de la página:
\pagestyle{fancy}
\rhead{\small{\asignatura}} % Nombre de la asignatura en mayúsculas
\lhead{\small{\leftmark}} % Nombre de la sección en minúsculas
% \rfoot{\small{Página \thepage}} % Número de página en el pie de página
\rfoot{\textbf{\color{main}\footnotesize{Página \thepage}}} % Número de página en el pie de página
% Cambio de color del número de página de la primera página:
\fancypagestyle{plain}{
    \fancyhf{}
    \cfoot{\textbf{\color{main}\footnotesize{\thepage}}} % Número de página en el pie de página
    \renewcommand{\headrulewidth}{0pt} % Quitar línea de la parte superior
    \renewcommand{\footrulewidth}{0pt} % Quitar línea de la parte inferior
    }
    \cfoot{} % Quitar número default de página:
    
    

% ------------------ ENCABEZADO DE DOCUMENTO ---------------------

\newcommand*{\FrontPage}[2]{%
    % Título, autor y fecha
    \maketitle

    % Espaciado entre párrafos
    \setstretch{1} % Espaciado

    % Incluir resumen si el primer argumento es 1
    \ifnum#1=1
        % Cambiar nombre del resumen a 'main' Resumen:
        \renewcommand{\abstractname}{\color{main}Resumen}
        \vspace{-2.5em}
        % \textcolor{main}{\rule{\textwidth}{0.4mm}}
        \begin{abstract}
            #2
        \end{abstract}
        \begin{center}
            \textcolor{main}{\rule{\textwidth}{0.4mm}}
        \end{center}
    \fi

    % \HRuleGray

    % Tabla de contenido:
    \begin{tcolorbox}[colback=main!5!white, colframe=main!75!black, title=\textbf{Tabla de Contenido}, toptitle=1.5mm, bottomtitle=1.5mm, breakable]
        % Quitar el título de la tabla de contenido, incluyendo el espacio que ocupa:
        \renewcommand{\contentsname}{}
        % Quitar el espacio entre el título y la tabla de contenido:
        \setlength{\cftbeforetoctitleskip}{-1cm}
        % Quitar el espacio entre el último ítem del índice y la línea:
        \setlength{\cftaftertoctitleskip}{1cm}
        % Esto requiere el paquete tocloft
        \tableofcontents
    \end{tcolorbox}
}




% ------------------------ DEFINICIONES --------------------------

\newcommand{\qed}{\par\noindent\hfill\ensuremath{\blacksquare}} % Define QED symbol that always appears on next line
\newcommand{\R}{\mathbb{R}}
\newcommand{\N}{\mathbb{N}}
\newcommand{\Z}{\mathbb{Z}} % Definir el conjunto de los enteros
\newcommand{\Q}{\mathbb{Q}} % Definir el conjunto de los racionales
\newcommand{\C}{\mathbb{C}} % Definir el conjunto de los complejos
\newcommand{\F}{\mathbb{F}} % Definir el conjunto de los números finitos
\newcommand{\T}{\mathbb{T}} % Definir el conjunto de los números reales
\newcommand{\E}{\mathbb{E}} % Esperado
\newcommand{\supA}{\sup(A)}
\newcommand{\infA}{\inf(A)}
\newcommand{\eps}{\varepsilon}



% ------------------- TÍTULO, AUTOR Y FECHA ----------------------

% Asignatura
\title{\begin{minipage}{\textwidth}
    \setstretch{0.8} % Espaciado
    \begin{center}
        \textbf{\Large{\textbf{\asignatura}}}
        % \HRuleGray
    \end{center}
\end{minipage}}

% Título del documento + autor
\author{\hspace*{-0.5em}\begin{minipage}{\textwidth}
\begin{center}
    % Título del documento + autor
    \setstretch{1.5} % Espaciado normal para el título
    % \textcolor{main}{\rule{1cm}{0.4mm}}\vspace{1em}\\
    \textbf{\LARGE{\color{main}\titulo}}\vspace{0.5em}\\\vspace{0.3em}
    \setstretch{1} % Espaciado normal para el título
    \autor
\end{center}
\end{minipage}}

% Fecha
\date{\today}
% \date{\today\\\vspace{1em}\textcolor{main}{\rule{1cm}{0.4mm}}}